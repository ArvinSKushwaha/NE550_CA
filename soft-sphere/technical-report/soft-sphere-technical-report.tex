\documentclass[10pt]{article}
\usepackage[utf8]{inputenc}
\usepackage[shortlabels]{enumitem}
\usepackage[margin=1in]{geometry}
\usepackage{derivative}
\usepackage{amsmath}
\usepackage{amsfonts}
\usepackage{fancyhdr}
\usepackage{graphicx}
\usepackage[dvipsnames]{xcolor}

\newcommand{\p}[1]{\left( #1 \right)}
\newcommand{\br}[1]{\left[ #1 \right]}
\newcommand{\abs}[1]{\left| #1 \right|}
\newcommand{\norm}[1]{\left\| #1 \right\|}
\newcommand{\s}[1]{\mathcal{#1}}
\newcommand{\bb}[1]{\mathbb{#1}}
\newcommand{\degC}{{}^\circ \text{C}}
\newcommand{\degF}{{}^\circ \text{F}}
\newcommand{\degK}{{}^\circ \text{K}}
\newcommand{\unit}[1]{\text{#1}}
\newcommand{\powp}[2]{\p{#1}^\p{#2}}
\newcommand{\pow}[2]{{#1}^{#2}}
\newcommand{\intlim}[4]{\int\limits_{#1}^{#2} #3 d #4}
% \newcommand{\choose}[2]{
% \begin{pmatrix}
%     #1 \\
%     #2
% \end{pmatrix}
% }


\newcommand{\studentname}{Arvin Kushwaha}
\newcommand{\teachername}{Dr. Eapen}
\newcommand{\coursenumber}{NE550}
\newcommand{\homeworkcount}{1}
\newcommand{\duedate}{February 18, 2022}

\usepackage{fancyhdr}

\pagestyle{fancy}
\fancyhf{}
\rhead{\coursenumber\ Homework \homeworkcount\ --- \duedate}
\lhead{\thepage\ \studentname}

\title{\coursenumber\ Computer Assignment \homeworkcount}
\author{\teachername\\ \studentname}
\date{\duedate}

\graphicspath{ {./} }
\setlength{\parindent}{0pt}

\begin{document}
\maketitle
\tableofcontents
\newpage

\section{Objective}

In this Computer Assigment, we aim to produce a molecular dynamics code using the C languageto use
the Lennard-Jones potential with a cutoff at $2^\frac{1}{6}\sigma$ to represent a soft-sphere
interaction. Using this code we investigate the temperature, potential and kinetic energy, as well
as total momentum, total energy, and pressure. Of these quantities, temperature and pressure are
thermodynamic properties that arise out of the statistical nature of energy and are only
well-defined for suffiiciently large systems.

\section{Approach}

To write this code, I use the C++ language, an extension onto the C language that allows additional
object-oriented features. The code consists of a struct representing 2D vectors along with the
respective operato overloads, and a struct to store the macrostates of our system. The class also
contains a class representing the system, which handles the initialization and updating of the
system, along with computation of the macrostates.

\section{Results}
\section{Discussion}
\section{Summary}

\end{document}
